\documentclass{article}
\usepackage{graphicx}
\usepackage{amsmath,amssymb,enumerate,graphicx,pgf,tikz,fancyhdr}
\usepackage{geometry}
\usepackage{tabvar}
\usepackage{fontspec}
\usepackage{dot2texi}

\usepackage{minted}
\usetikzlibrary{backgrounds}
\usetikzlibrary{arrows.meta}
\usetikzlibrary{shapes.geometric}

\title{\centering Majeure Informatique: 
Jeu d'échecs}

\author{BOYER Timothé, MOURET Basile, HACINI Malik}
\date{26 Septembre 2023}
\renewcommand{\contentsname}{Table des Matières}

\renewcommand{\theFancyVerbLine}{
    \sffamily\textcolor[rgb]{0.5,0.5,0.5}{\scriptsize\arabic{FancyVerbLine}}}
    
    
\begin{document}
    
    
\csundef{listing}\csundef{endlisting}
\csundef{listing*}\csundef{endlisting*}

\maketitle
\tableofcontents{}

\section{Introduction}
L’objectif de ce projet est de programmer (et tester) un jeu
d’échecs qui permet à deux joueurs de s’affronter, 
chaque joueur peut être soit un humain, soit l’ordinateur via une IA
 (intelligence artificielle) . Le jeu se jouera dans un terminal, 
 et si les deux joueurs sont humains, ils utiliseront le même clavier.
\section{Conception}
\subsection{Règles traitées}
traitées: promotion, roque
non traitées : en passant
\subsection{Architecture}
Ce projet a été réalisé en utilisant les outils de la programmation orientée objet (POO).ù
Les différentes composantes d'un jeu d'échecs (joueurs, pieces) sont donc 
représentées par des classes.
\subsubsection{Diagramme de classe UML}
a mettre



\section{implémentation}
\section{tests}
\section{peut être un de plus jsp}
\end{document}